\documentclass[10pt,a4paper]{article}
\usepackage[latin1]{inputenc}
\usepackage{amsmath}
\usepackage{amsfonts}
\usepackage{amssymb}
\usepackage{fullpage}

\begin{document}
\title{J.D. Jackson Problem 5.6}
\author{Josh Orndorff \\ admin@joshorndorff.com}
\maketitle

To begin I'll find the $B$-field due to a cylindrical conductor of radius $a$ that does \textit{not} have a hole bored in it.
\begin{equation}
\oint \mathbf{B_a} \cdot\mathrm{d}\mathbf{l} = \mu_0I_{enc}
\end{equation}

We know by symmetry that the $B$-field would be equal in magnitude, so it can come out of the integral, and we obtain a familiar result.
\begin{equation}
B_a=\frac{\mu_0}{2}jr
\end{equation}

In full vector notation, we use the right hand rule to obtain
\begin{equation}
\mathbf{B_a}=\frac{\mu_0}{2}(\mathbf{j}\times\mathbf{r})
\end{equation}

By the same logic we see that the $B$-field from \textit{only} a conducting cylinder of radius $b$ a distance $d$ from the origin is
\begin{equation}
\mathbf{B_b}=\frac{\mu_0}{2}(\mathbf{j}\times(\mathbf{r}-\mathbf{d}))
\end{equation}

Now the critical step is that the actual $B$-field in the hole is given by
\begin{align}
\mathbf{B_{tot}}&=\mathbf{B_a}-\mathbf{B_b} \\
&=\frac{\mu_0}{2}[\mathbf{j}\times\mathbf{r}-\mathbf{j}\times(\mathbf{r}-\mathbf{d})] \\
&=\frac{\mu_0}{2}\mathbf{j}\times[\mathbf{r}-\mathbf{r}+\mathbf{d}]
\end{align}

\begin{equation}\boxed{
\mathbf{B_{tot}}=\frac{\mu_0}{2}(\mathbf{j}\times\mathbf{d})
}\end{equation}
\end{document}