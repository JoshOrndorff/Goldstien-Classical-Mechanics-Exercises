\documentclass[10pt,a4paper]{article}
\usepackage[latin1]{inputenc}
\usepackage{amsmath}
\usepackage{amsfonts}
\usepackage{amssymb}
\usepackage{fullpage}
\usepackage{graphicx}
\usepackage{parskip}

\begin{document}
\title{J.D. Jackson Problem 1.9}
\author{Josh Orndorff \\ admin@joshorndorff.com}
\maketitle

\section{Parallel Plates}
We have seen several times (e.g. problem 1.6) that the electric field due to an infinite sheet of surface charge can be calculate with a Gaussian pillbox.
\begin{equation}
\mathbf{E}=\frac{\sigma}{2\epsilon_0}\mathbf{\hat{n}}
\end{equation}

The force per unit area on the other plate (the negative plate) is as follows. Remember that $\mathbf{\hat{n}}$ points away from the first plate, which means the force is attractive.
\begin{equation}
\frac{\mathbf{F}}{A}=-\sigma \mathbf{E} = \frac{-\sigma^2}{2\epsilon_0}\mathbf{\hat{n}}
\end{equation}

If we want an expression in terms of constant voltage rather than charge density, we only need to substitute $\sigma=\epsilon_0 V / d$ which we found in problem 1.6 to get
\begin{equation}
\frac{\mathbf{F}}{A}=\frac{-\epsilon_0 V^2}{2d^2}\mathbf{\hat{n}}
\end{equation}

The force is still attractive.

\section{Parallel Cylinders}
The parallel cylinder case is almost identical. We found the electric field of one cylinder to be,
\begin{equation}
\mathbf{E}=\frac{\lambda}{2\pi r\epsilon_0}\mathbf{\hat{r}}
\end{equation}

We are told that the cylinder separation $d$, is much greater than either radius, so we can say that over the small cross section of the second cylinder, $\mathbf{E}$ is roughly constant and that,
\begin{equation}
\frac{\mathbf{F}}{l}=-\lambda\mathbf{E}=\frac{-\lambda^2}{2\pi d \epsilon_0}\mathbf{\hat{r}}
\end{equation}

The approximation will be better as the ratio $d/a$ increases. If you are uncomfortable with approximating the result, the objection should be raised upstream because we already used it in calculating capacitance in problem 1.7.

To get the result in terms of electric potential, we just substitute $\lambda = \pi\epsilon_0 V/\ln(a/d)$.
\begin{equation}
\frac{\mathbf{F}}{l}=\frac{-\pi\epsilon_0 V^2}{2d\ln^2(a/d)}
\end{equation}

\end{document}