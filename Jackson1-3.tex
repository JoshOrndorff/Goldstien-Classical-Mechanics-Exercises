\documentclass[10pt,a4paper]{article}
\usepackage[latin1]{inputenc}
\usepackage{amsmath}
\usepackage{amsfonts}
\usepackage{amssymb}
\usepackage{fullpage}
\usepackage{graphicx}
\usepackage{parskip}

\begin{document}
\title{J.D. Jackson Problem 1.3}
\author{Josh Orndorff \\ admin@joshorndorff.com}
\maketitle

\section{Spherical Shell}
Although we don't have a choice in this problem, it is wise to use spherical coordinates because of the geometry's spherical symmetry.

We know by symmetry that $\rho(\mathbf{x}) = \rho(r)$. We also know that there is only charge at a specific radius, $R$, and that $\rho$ should be proportional to the total charge $Q$. Let's call our proportionality constant $A$.

\begin{equation}\label{a-density-1}
\rho(r)=AQ\delta(r-R)
\end{equation}

In order to find the proportionality constant, $A$, we need the charge density's integral over all space to return the total charge, $Q$.
\begin{equation}
\iiint\rho(r) \mathrm{d}^3x = Q
\end{equation}

\begin{equation}
\int_0^\pi \int_0^{2\pi}\int_0^\infty AQ\delta(r-R) \,r^2 \sin\theta \,\mathrm{d}r\, \mathrm{d}\phi\, \mathrm{d}\theta
\end{equation}

Performing the angular integrals is trivial, and we can remove the constants $A$, and $Q$ from the integral.
\begin{equation}
4\pi AQ\int_0^\infty \delta(r-R)\,r^2\,\mathrm{d}r =Q
\end{equation}

Cancelling the $Q$'s and performing the integral gives
\begin{equation}
4\pi AR^2=1
\end{equation}
\begin{equation}
A=\frac{1}{4\pi R^2}
\end{equation}

Finally substituting back into equation \ref{a-density-1} gives
\begin{equation}\boxed{
\rho(r) = \frac{Q}{4\pi R^2}\delta(r-R)
}\end{equation}

As a final sanity check, we confirm that the units are Coulombs per cubic meter, which they are (remember the units on a delta function).  Note also, that the $R^2$ could be replaced with $r^2$ as integrating over the delta function will yield the same result in either case.

\section{Cylindrical Shell}
We'll start as before by observing cylindrical symmetry, and using a proportionality constant.
\begin{equation}
\rho(r) = A\lambda \delta(r-b)
\end{equation}

Again we'll integrate, but this time we will not integrate over all space. The total charge on the cylinder is infinite because it stretches to infinity in both $z$-directions. Rather we will integrate in a plane perpendicular to that axis to get the specified linear charge density.
\begin{equation}
\int_0^{2\pi}\int_0^\infty A\lambda\delta(r-b)\, r \,\mathrm{d}r\, \mathrm{d}\theta = \lambda
\end{equation}

Performing the radial integral and taking constants outside the integral,
\begin{equation}
2\pi A\lambda\int_0^\infty \delta(r-b)\,r\,\mathrm{d}r = \lambda
\end{equation}

Doing the final integral, canceling $\lambda$, and solving for $A$,
\begin{equation}
A=\frac{1}{2\pi b}
\end{equation}

So the final result is
\begin{equation}\boxed{
\rho(r)=\frac{\lambda}{2\pi b}\delta(r-b)
}\end{equation}

\section{Flat Disc (Cylindrical Coordinates)}
Although Jackson doesn't mention it, we will need to use Heaviside step functions for this one to terminate the disc.
\begin{equation}
\rho(r, z)=AQ\delta(z)\Theta(R-r)
\end{equation}

\begin{equation}
\int_{-\infty}^{\infty}\int_0^\infty\int_0^{2\pi}AQ\delta(z)\Theta(R-r)\,r\,\mathrm{d}\theta\,\mathrm{d}r\,\mathrm{d}z = Q
\end{equation}

\begin{equation}
2\pi AQ\int_0^\infty \Theta(R-r)\,r\,\mathrm{d}r = Q
\end{equation}

\begin{equation}
2\pi A\int_0^R r\,\mathrm{d}r = 1
\end{equation}

\begin{equation}
A=\frac{1}{\pi R^2}
\end{equation}

\begin{equation}\boxed{
\rho(r, z)=\frac{Q}{\pi R^2}\delta(z)\Theta(R-r)
}\end{equation}

\section{Flat Disc (Spherical Coordinates)}
Again, we will need to use Heaviside step functions to terminate the disc.
\begin{equation}
\rho(r, z)=AQ\delta\left(\theta-\frac{\pi}{2}\right)\Theta(R-r)
\end{equation}

\begin{equation}
\int_0^\infty\int_0^\pi\int_0^{2\pi}AQ\delta\left(\theta-\frac{\pi}{2}\right)\Theta(R-r)\,r^2\sin\theta\,\mathrm{d}\phi\,\mathrm{d}\theta\,\mathrm{d}r\ = Q
\end{equation}

Perform the $\phi$ integral and take care of that Heaviside function by changing the bounds on the $R$ integral.
\begin{equation}
2\pi AQ\int_0^R\int_0^\pi\delta\left(\theta-\frac{\pi}{2}\right)\,r^2\sin\theta\,\mathrm{d}\theta\,\mathrm{d}r\ = Q
\end{equation}

Cancel the $Q$'s and do the theta integral.
\begin{equation}
2\pi A\int_0^R r^2\,\mathrm{d}r\ = 1
\end{equation}

Do the final integral and solve for $A$.
\begin{equation}
A=\frac{3}{2\pi R^3}
\end{equation}

Plugging back into the original charge density
\begin{equation}\boxed{
\rho(r, z)=\frac{3Q}{2\pi R^3}\delta\left(\theta-\frac{\pi}{2}\right)\Theta(R-r)
}\end{equation}


\end{document}