\documentclass[10pt,a4paper]{article}
\usepackage[latin1]{inputenc}
\usepackage{amsmath}
\usepackage{amsfonts}
\usepackage{amssymb}
\usepackage{fullpage}

\begin{document}
\title{J.D. Jackson Problem 4.6}
\author{Josh Orndorff \\ admin@joshorndorff.com}
\maketitle

\section{Energy of quadrupole interaction}
The correct solution of this problem relies heavily on little factoids that Jackson drops on us with little justification.  First he tells us that $\rho$ is cylindrically symetric.  Second he provides an external field that is cylindrically symetric about the same axis.  Third he define $Q\equiv\frac{1}{e}Q_{33}$.  So for this part there isn't a whole lot left to do.

We start with the quadripole term of the work equation (4.24) in the text.
\begin{equation}
W=-\frac{1}{6}\sum_1\sum_j Q_{ij} \frac{\partial E_j}{\partial x_i}(0)
\end{equation}

Another little nugget of truth that Jackson gives us is that all the $Q_{ij}$'s are zero except for $Q_{33}$.  But, importantly, he footnotes on page 151 that really all the diagonal terms are nonzero such that $Q_{11}=Q_{22}=-\frac{1}{2}Q_{33}$.  We can also use Gauss's law for electrostatics to show that,
\begin{equation}
\nabla\cdot \mathbf{E} = \mathbf{0}
\end{equation}
\begin{equation}
\frac{\partial E_x}{\partial x} + \frac{\partial E_y}{\partial y} + \frac{\partial E_z}{\partial z} = 0
\end{equation}
\begin{equation}
\frac{\partial E_x}{\partial x} = \frac{\partial E_y}{\partial y} = -\frac{1}{2}\frac{\partial E_z}{\partial z}
\end{equation}

So the work equation can be written more simply.
\begin{equation}
W=\frac{1}{6}\left[
\frac{-1}{2}Q_{33}\frac{-1}{2}\frac{\partial E_z}{\partial z}+
\frac{-1}{2}Q_{33}\frac{-1}{2}\frac{\partial E_z}{\partial z}+
Q_{33}\frac{\partial E_z}{\partial z}
\right]
\end{equation}

\begin{equation}
W=\frac{-1}{6}Q_{33}\frac{\partial E_z}{\partial z} \left[\frac{1}{4}+\frac{1}{4}+1 \right]
\end{equation}

\begin{equation}
\frac{-1}{4}Q_{33}\frac{\partial E_z}{\partial z}
\end{equation}

Substituting the definition of quadripole moment,
\begin{equation}
\frac{-e}{4}Q\frac{\partial E_z}{\partial z}
\end{equation}

\section{An example with numbers}
I didn't do this part.  Maybe I'll get back to it eventually.

\section{Elipsoidal nucleus model}
Starting with equation (4.9) or (4.25),
\begin{equation}
Q=\frac{1}{e}\int 3z^2-\left(\sqrt{x^2+y^2+z^2}\right)^2 \rho(\mathbf{x})\, \mathrm{d}V
\end{equation}

We know that $\rho$ is constant from the problem, so its value is $\rho=\frac{Ze}{V}$  Where the volume of the elipsoid is $V=\frac{4}{3}\pi ab^2$.  We should note here that I have chosen a single major axis oriented along the $z$-axis.  It would have been equally valid to choose a single minor axis along the $z$-axis and two major axes in the $xy$-plane.
\begin{equation}
Q=\frac{3Z}{4\pi ab^2}\int 2z^2-x^2-y^2 \, \mathrm{d}V
=\frac{3Z}{4\pi ab^2}\int 2z^2-r^2 \, \mathrm{d}V
\end{equation}

See the final section for how to get the bounds of integration.
\begin{equation}
Q=\frac{3Z}{4\pi ab^2} \int_{-a}^a\int_0^{b\sqrt{1-\frac{z^2}{a^2}}}\int_0^{2\pi}
2z^2r-r^3 \, \mathrm{d}\theta \, \mathrm{d}r \, \mathrm{d}z
\end{equation}

Performing the first two integrations,
\begin{equation}
Q=\frac{3Z}{2 ab^2} \int_{-a}^a
\left. z^2r^2-\frac{1}{4}r^4 \right|_0^{b\sqrt{1-\frac{z^2}{a^2}}} \,\mathrm{d}z
\end{equation}

\begin{equation}
Q=\frac{3Z}{2 ab^2} \int_{-a}^a z^2b^2 \left(1-\frac{z^2}{a^2}\right)-\frac{1}{4}b^4\left(1-\frac{z^2}{a^2}\right)^2 \,\mathrm{d}z
\end{equation}

\begin{equation}
Q=\frac{3z}{2a}\int_{-a}^a z^2-\frac{z^4}{a^2}-\frac{b^2}{4}+\frac{b^2z^2}{2a^2}-\frac{b^2z^4}{4a^4}\,\mathrm{d}z
\end{equation}

\begin{equation}
Q=\frac{3z}{a}\left[\frac{-b^2a}{4}+\frac{a^3}{3}+\frac{b^2a}{6}-\frac{a^3}{5}-\frac{b^2a}{20}\right]
\end{equation}

\begin{equation}
Q=\frac{Z}{20}(-8b^2+8a^2)
\end{equation}

\begin{equation}
\boxed{
Q=\frac{2Z}{5}(a^2-b^2)
}
\end{equation}

Next Jackson wants us to solve for the factor $\frac{a-b}{R}$.
\begin{equation}
Q=\frac{4Z}{5}(a-b)\frac{a+b}{2}=\frac{4Z}{5}(a-b)R=\frac{4Z}{5}\frac{a-b}{R}R^2
\end{equation}

Solving for the desired factor,
\begin{equation}
\frac{a-b}{R}=\frac{5Q}{4ZR^2}
\end{equation}

And plugging in the given values,
\begin{equation}
\frac{a-b}{R}=\frac{5\cdot(2.5 E^{-28})}{4\cdot 63\cdot(7.0 E^{-15})^2} \approx0.10123
\end{equation}

\appendix
\section{Bounds of integration}
It's well established (for example check out wikipedia) that the equation for an elipsoid aligned along the $z$-axis is
\begin{equation}
\frac{x^2+y^2}{b^2}+\frac{z^2}{a^2}=1
\end{equation}

We want to integrate over the volume of the elipsoid, so the $theta$ inegral is easy enough.  We just go all the way around the $xy$-plane.  I'll choose to make the $z$ integral easy too by integrating from $-a$ to $a$.  This means that all we need to do is solve the elipsoid equation for $r$ to get the bounds for $z$.
\begin{equation}
\frac{r^2}{b^2}=1-\frac{z^2}{a^2}
\end{equation}
\begin{equation}
r=b\sqrt{1-\frac{z^2}{a^2}}
\end{equation}

So we will integrate the $r$ coordinate from $0$ to the critical $r$ value we just found.

\end{document}