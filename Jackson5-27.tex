\documentclass[10pt,a4paper]{article}
\usepackage[latin1]{inputenc}
\usepackage{amsmath}
\usepackage{amsfonts}
\usepackage{amssymb}
\usepackage{fullpage}

\begin{document}
\title{J.D. Jackson Problem 5.27}
\author{Josh Orndorff \\ admin@joshorndorff.com}
\maketitle

We'll begin by finding the magnetic field in the three regions.  This calculation is somewhat simple because the problem's symmetry allows the use of Ampere's law.
\begin{align}
\mathbf{B_{in}}&=\frac{\mu Ir}{2\pi b^2}\mathbf{\hat{\phi}} \\
\mathbf{B_{mid}}&=\frac{\mu_0 I}{2\pi r}\mathbf{\hat{\phi}} \\
\mathbf{B_{in}}&=\mathbf{0}
\end{align}

By equation (5.148) and a simplified version of (5.152) we see that
\begin{equation}
L=\frac{1}{I^2}\int\frac{B^2}{\mu}\,\mathrm{d}V
\end{equation}

\begin{equation}
L=\frac{1}{I^2}\int_0^{2\pi}\int_0^l \left[\int_0^b\frac{\mu_0^2I^2r^2}{4\pi^2\mu b^4}r\,\mathrm{d}r + \int_b^a\frac{\mu_0^2 I^2}{4\pi^2\mu_0 r^2}r\,\mathrm{d}r +0\right]\,\mathrm{d}z\mathrm{d}\theta
\end{equation}

\begin{equation}\boxed{
\frac{L}{l}=\frac{\mu_0^2}{2\pi}\left[\frac{1}{4\mu}+\frac{1}{\mu_0} \ln\frac{a}{b}\right]
}\end{equation}


In the case that the inner conductor is hollow, the $B$-field in the inner-most region becomes zero, and the Inductance per unit length simplifies
\begin{equation}\boxed{
\frac{L}{l}=\frac{\mu_0}{2\pi}\ln\frac{a}{b}
}\end{equation}

If we return to the more common definition of inductance (\textbf{not} per unit length) we get
\begin{align}
\xi&=-\frac{\mathrm{d}}{\mathrm{d}t}\iint\mathbf{B}\cdot\mathrm{d}\mathbf{A} \\
&=-l\frac{\mathrm{d}}{\mathrm{d}t}\int_b^a\frac{\mu I}{2\pi r}\,\mathrm{d}r \\
&=-\frac{l\mu_0}{2\pi}\frac{\mathrm{d}I}{\mathrm{d}t}\ln\frac{a}{b}
\end{align}

\begin{equation}\boxed{
L=\frac{-\xi}{\frac{\mathrm{d}I}{\mathrm{d}t}}=\frac{l\mu_0}{2\pi}\ln\frac{a}{b}
}\end{equation}

\end{document}