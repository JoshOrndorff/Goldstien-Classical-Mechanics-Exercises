\documentclass[10pt,a4paper]{article}
\usepackage[latin1]{inputenc}
\usepackage{amsmath}
\usepackage{amsfonts}
\usepackage{amssymb}
\usepackage{fullpage}

\begin{document}
\title{Goldstein, Poole, and Safko Problem 3.29}
\author{Josh Orndorff \\ admin@joshorndorff.com}
\maketitle

We set up the coordinate system as shown with the z-axis coming out of the page.

Goldstein dictates that we begin by calculating the cross product $\mathbf{L}\times\mathbf{A}$. So, using the definition of $\mathbf{A}$ from equation 3.82 we we have the following.

\begin{equation}
\mathbf{L}\times\mathbf{A} = \mathbf{L} \times (\mathbf{p}\times \mathbf{L}-mk\mathbf{\hat{r}}) \\
\end{equation}
\begin{equation}
\mathbf{L}\times\mathbf{A} = \mathbf{L} \times (\mathbf{p} \times \mathbf{L}) - \mathbf{L} \times mk \mathbf{\hat{r}}
\end{equation}
Using the double curl vector identity, this becomes:
\begin{equation}
\mathbf{L}\times\mathbf{A} = \mathbf{p}(\mathbf{L} \cdot \mathbf{L}) - \mathbf{L}(\mathbf{L} \cdot \mathbf{p}) - \mathbf{L} \times mk \mathbf{\hat{r}}
\end{equation}
Each term in the previous equation simplifies. On the left side, $\mathbf{L}\times\mathbf{A}=-lA\mathbf{\hat{y}}$. In the first term on the right, $\mathbf{L} \cdot \mathbf{L} = l^2$. The second term is zero because $\mathbf{L}$ and $\mathbf{p}$ are perpendicular. And evaluating the cross product in the third term gives $\mathbf{L} \times mk \mathbf{\hat{r}} = mkl\mathbf{\hat{\theta}}$.
\begin{equation}
-lA\mathbf{\hat{y}} = \mathbf{p}l^2 - mkl \mathbf{\hat{\theta}}
\end{equation}
Now we'll rearrange and take the magnitude of both sides.
\begin{equation}
\left|\mathbf{p}+\frac{A}{l} \mathbf{\hat{y}}\right|=\frac{mk}{l}
\end{equation}

The form of the equation above shows that the hodograph traces out a circle of radius $mk/l$. The explicit addition of the $\mathbf{\hat{y}}$ term on the left shows that the circle is shifted along the y axis by a distance $A/l$.

\end{document}